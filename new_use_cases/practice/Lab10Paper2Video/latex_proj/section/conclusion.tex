\section{Conclusions}

\textbf{litstudy} provides a comprehensive solution for researchers who need to explore and understand scientific literature in their domain. The package's modular architecture, extensive functionality, and ease of use make it an invaluable tool for literature review and bibliometric analysis.

The package's success in real-world applications demonstrates its practical value and potential for widespread adoption in the research community. Future development will focus on expanding the supported data sources, improving the NLP capabilities, and enhancing the visualization features.

Key contributions of this work:

\begin{itemize}
\item \textbf{Unified Interface}: Provides a single interface for accessing multiple bibliographic data sources.
\item \textbf{Comprehensive Analysis}: Supports various types of analysis including statistics, networks, and topic modeling.
\item \textbf{User-Friendly}: Easy to use from Python scripts and Jupyter notebooks.
\item \textbf{Extensible}: Modular design allows for easy extension and customization.
\item \textbf{Well-Documented}: Comprehensive documentation and examples for users.
\end{itemize}

The \textbf{litstudy} package represents a significant contribution to the field of bibliometric analysis and literature review, providing researchers with powerful tools to explore and understand scientific literature in their domains.
