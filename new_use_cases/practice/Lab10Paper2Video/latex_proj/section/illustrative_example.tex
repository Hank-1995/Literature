\section{Illustrative example}

To demonstrate the capabilities of \textbf{litstudy}, we provide a comprehensive example that shows how to use the package for exploring a research domain. The example covers:

\begin{itemize}
\item Data collection from multiple sources
\item Data cleaning and preprocessing
\item Statistical analysis and visualization
\item Network analysis
\item Topic modeling
\end{itemize}

The following code snippet demonstrates the basic usage of \textbf{litstudy}:

\begin{verbatim}
import litstudy

# Create a new study
study = litstudy.Study()

# Add documents from different sources
study.add_documents_from_scopus(query="machine learning")
study.add_documents_from_semantic_scholar(query="deep learning")

# Filter and clean the data
study.filter_by_year(min_year=2020)
study.remove_duplicates()

# Generate statistics
stats = study.compute_statistics()
study.plot_publications_per_year()

# Create bibliographic networks
network = study.create_citation_network()
study.plot_network(network)

# Perform topic modeling
topics = study.discover_topics()
study.plot_topic_distribution(topics)
\end{verbatim}

This example shows how \textbf{litstudy} can be used to quickly explore a research domain, from data collection to analysis and visualization.
