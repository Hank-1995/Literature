\documentclass{beamer}


\usetheme{Madrid}
\usepackage{amsmath}
\usepackage{amsfonts}
\usepackage{amssymb}
\usepackage{graphicx}
\usepackage{booktabs}
\usepackage{hyperref}
\usepackage{xcolor}

\title{litstudy: A Python package for literature reviews}
\author{Stijn Heldens, Alessio Sclocco, Henk Dreuning, Ben van Werkhoven, Pieter Hijma, Jason Maassen, Rob V. van Nieuwpoort}
\date{\today}

\setbeamerfont{caption}{size=\scriptsize}


\setbeamerfont{caption}{size=\scriptsize}
\begin{document}
\begin{frame}
\titlepage
\end{frame}
\begin{frame}
\frametitle{Motivation and Significance}
\begin{itemize}
\item Researchers often need to explore new scientific domains outside their field of expertise.
\item Getting a broad overview of an area of study is difficult due to the large number of relevant publications.
\item Literature reviews present a solution, but they might not always be available, may be too broad or too narrow in scope, and can be outdated quickly.
\end{itemize}
\end{frame}
\begin{frame}
\frametitle{Related Work}
\begin{itemize}
\item Bibliometric analysis and literature review are essential tools for researchers.
\item Current tools and methods have limitations, such as being too broad or too narrow in scope, or being outdated quickly.
\end{itemize}
\end{frame}
\begin{frame}
\frametitle{Method}
\begin{itemize}
\item litstudy: a Python package for literature reviews.
\item Modular architecture, extensive functionality, and ease of use.
\item Five main features:
\begin{itemize}
\item Extract metadata of scientific documents from various sources.
\item Filter, select, deduplicate, and annotate collections of documents.
\item Compute and plot general statistics on the metadata of the documents.
\item Generate, plot, and analyze various bibliographic networks that reveal relations between publications and their authors.
\item Automatic topic discovery based on natural language processing (NLP).
\end{itemize}
\end{itemize}
\end{frame}
\begin{frame}
\frametitle{Method - Bibliographic Data Sources}
\begin{itemize}
\item Supports several methods to retrieve metadata of scientific publications.
\item Unified interface allows data from different sources to be combined.
\end{itemize}
\end{frame}
\begin{frame}
\frametitle{Method - Filtering}
\begin{itemize}
\item Functionality to filter, select, deduplicate, and annotate collections of documents.
\item Essential for managing large datasets and ensuring data quality.
\end{itemize}
\end{frame}
\begin{frame}
\frametitle{Method - Statistics}
\begin{itemize}
\item Computes and plots general statistics on the metadata of the documents.
\item Includes statistics per year, per author, per journal, and other relevant metrics.
\end{itemize}
\end{frame}
\begin{frame}
\frametitle{Method - Bibliographic Networks}
\begin{itemize}
\item Generates, plots, and analyzes various bibliographic networks that reveal relations between publications and their authors.
\item Helps identify influential papers, authors, and research clusters.
\end{itemize}
\end{frame}
\begin{frame}
\frametitle{Method - Natural Language Processing}
\begin{itemize}
\item Enables automatic topic discovery based on natural language processing techniques.
\item Helps researchers identify key themes and trends in their research domain.
\end{itemize}
\end{frame}
\begin{frame}
\frametitle{Experimental Method}
\begin{itemize}
\item Comprehensive example demonstrating the capabilities of litstudy.
\item Data collection from multiple sources, data cleaning and preprocessing, statistical analysis and visualization, network analysis, and topic modeling.
\end{itemize}
\end{frame}
\begin{frame}
\frametitle{Experimental Setting}
\begin{itemize}
\item Dataset: multiple sources of scientific publications.
\item Parameters: various parameters for data cleaning, filtering, and analysis.
\item Environment: Python, litstudy package, and Jupyter notebooks.
\end{itemize}
\end{frame}
\begin{frame}
\frametitle{Experimental Results}
\begin{itemize}
\item Successful demonstration of litstudy's capabilities.
\item Results show the effectiveness of litstudy in exploring a research domain.
\end{itemize}
\end{frame}
\begin{frame}
\frametitle{Ablation Experiment}
\begin{itemize}
\item Validation of the role of key modules in litstudy.
\item Results show the importance of each module in the overall functionality of litstudy.
\end{itemize}
\end{frame}
\begin{frame}
\frametitle{Deficiencies}
\begin{itemize}
\item Limitations of current methods and tools for literature review and bibliometric analysis.
\item litstudy addresses some of these limitations, but there is still room for improvement.
\end{itemize}
\end{frame}
\begin{frame}
\frametitle{Future Research}
\begin{itemize}
\item Expanding the supported data sources.
\item Improving the NLP capabilities.
\item Enhancing the visualization features.
\end{itemize}
\end{frame}
\begin{frame}
\frametitle{Thank you}
\end{frame}
\end{document}